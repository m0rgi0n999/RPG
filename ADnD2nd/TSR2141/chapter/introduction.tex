



== Introduction ==

When the DUNGEONS & Dracons* Game first appeared in 1973, no one was sure how the public would receive it. Nothing could prepare the fledgling Tactical Studies Rules for the stir that D&D* created in the hearts and minds of millions of fantasy lovers all over the world. Twenty years later, fantasy role-playing, the ADVANCED DUNGEONS & DRaGons® and the DUNGEONS & DRAGONS games are better than ever, and there is no end in sight.

The uniqueness of these two game systems lies in the fact that they are open-ended. The game constantly changes and continually expands with new monsters, extra spells for spellcasters, supplementary proficiencies, and supernumerary magical items—there is nothing that can’t be added to the AD&D®* game system.

Throughout the 19-year history of the D&D and AD&D game worlds, reams of paper and gallons of ink have detailed thousands of magical items. From the original D&D boxed set and the first issue of THE STRATEGIC REVIEW Magazine, to the last products shipped in December 1993, almost every product has featured at least one new magical item, and you will find them all in the ENCYCLOPEDIA MAGICA™ collection. Since many of these products are long out of print, it is impossible for the majority of players and Dungeon Masters around the world to enjoy these unique treasures. We had to call a halt at some point, so these volumes only include those products shipped through December 1993.

Besides the magical items from game products, we have included those from Dragon® Magazine, up to #199; DunGEon* Magazine, issues 1 through 45; all 30 issues of IMAGINE Magazine; PoLYHEDRON*® Newszines through 90, and all seven issues of THE STRATEGIC REVIEW.

In 1991 and 1993, TSR produced a two-volume set, The Magical Encyclopedia, The older encyclopedia is an index of magical items and where these PC trophies can be found among the myriad TSR products. Unfortunately, most of those older products are unavailable—many are now collectors’ items.

We decided to do something about that. The ENCYCLOPEDIA MAGICA volumes feature every magical item we could find, in every product we have ever created. From obscure references to

special weapons in the first modules to the detailed weapons of the DUNGEON MASTER® Guide (DMG), players and DMs alike will have years of enjoyment discovering the thousands of items enchanted by the wonderful power of magic.

DMs are forewarned though, just because an item is listed here does not mean it should be given to players casually. Many items are artifacts (often believed to be items of extreme power), relics (items of historic or sacred value), or items of such potency they can destroy a continuing campaign. Be very careful when releasing these items to your players.

With the enormous number of magical items woven into game systems, it is difficult to overcome the attitude that magic is everywhere and easy to obtain. Given the amount of time, energy, and self-sacrifice required to produce even a dagger +0, finding a single magical item should be a marvelous event in the lives of your characters. (See the Book of Artifacts for information on magical item creation.)

It is not surprising then that the earliest magical items (especially weapons) were given names. From Bucknard'’s everfull purse to the Equalizer (an extremely lethal sword), rare items were named for the individuals who created them or who first used them heroically. However they came by their titles, they now belong to the realm of legend.

Players should be encouraged by the DM to name the items they possess, or the DM should name them before granting them to a player. After all, finding Prismal’s wand of lightning with 9 charges remaining is much more exciting than stumbling over a generic wand of lightning with 10 charges, especially when the players learn that Prismal’s wand overwhelmed the chateau of a renowned and virtuous noble. Players will enjoy items of historical significance more than mundane ones, and a note of realism will be added to your campaign. Remember too, that items themselves can become characters.

ALWAYS AN ADVENTURE!™ is the basic premise for every product produced by TSR, and this encyclopedia is no exception. The ENCYCLOPEDIA MasIca collection is designed to be a useful and enjoyable addition to any role-player’s library, one that will stimulate years of adventure.

=== How to Use These Books ===

The last volume of this encyclopedia contains tables that randomly determine which magical items are found in a treasure hoard, though artifacts and most relics are excluded from the tables. (For those players and DMs who prefer to play a tongue-in-cheek game, we included pun items in the random tables.) We suggest that DMs select the items PCs discover in order to keep garme-busters from appearing at random.

Each item listed in the ENCYCLOPEDIA MAGICA collection marks its type and name, gives the experience point and gold piece values, the source, and a detailed description. To find a particular item, look under its name or use the detailed index supplied in the final volume.

Magical Item Name: Most often, this is the name given to the item in its first (and usually) subsequent appearances. Unfortunately, some names changed. This was done for three reasons. First, we tried to make items more useful, exciting, or easier to find. Second, a few were changed in order to make placement more convenient and logical. Finally, some of the D&D and first edition items had names that were changed in later editions, updates, and product lines. These were brought into line with current usage to prevent confusion.

In many cases, items were grouped together to create order and make it easier to find things. For example, short swords, bastard swords, and all other swords are grouped together under the single entry Sword. This means that a DM looking for information on the sword of dancing doesn’t have to remember if it’s a short sword or a long sword. These entries typically have a random table at the beginning of the entry that allows the Dungeon Master to create a larger repertoire of magical items. (The Sword entry’s random table, for example, includes rapier, bastard sword, gladius, claymore, khopesh, cutlass, ninja-to, among many, many others.)

Experience Point Value: To make use of an item’s experience point value, check the particular set of rules you are using. In the original AD&D game, experience is awarded only for items kept and used on an adventure. This helped to check the idea that killing monsters and NPCs is the only way to gain experience. In the AD&D 1nd Edition game, experience is awarded to the

character who creates an item in order to slow level progression, In the D&D game, experience points are not usually awarded for magical items. However, many Dungeon Masters find it convenient to grant experience points to characters who find and keep items no matter what rules the campaign uses.

In some entries, experience point value is listed as “—”. This indicates a unique and potent item (often called an artifact or relic). Artifacts and relics have powers far beyond what even the most heroic mortal can wield without detrimental effects to his or her health and psyche. Experience is never given for these items, and the DM must carefully monitor their use in a campaign. Generally, it is unwise to introduce any magical item into a game unless the DM understands the item’s powers and how those powers affect play—this is especially true for artifacts and relics.

Gold Piece Value: Throughout the ENCYCLOPEDIA MAGICA volumes, items are given a value, listed in gp. This does not mean that every item can be purchased by simply surrendering the specified coinage. Instead, gp value is used to rate the relative values of different items to each other. In individual campaigns, the Dungeon Master may wish to increase the “bartering value” of items two-fold, ten-fold, or even 99-fold to keep magical items the rare and wonderful things that they are, and it’s a great way to keep the campaign from overbalancing. See the section below on Monetary System Conversions for an extensive table to convert the AD&D gold piece standard to any AD&D campaign world that uses a different standard. Many entries have a gold piece value rated “—”. This means that the item is so priceless that it is impossible to rate its true barter value.

First Appearance or Best Reference: The small type (DRAGON Magazine 1) designates either the first source in which an item appeared, or the title of the most complete reference.

Description/Explanation: This is the descriptive text of an item. Often, the original passage has been altered or expanded to clarify the use of the item or its history. In a few cases, the description is either significantly or completely changed (but this is rare).

Running Magical Item Total: Along the footer of each page is a series of numbers. These enumerate the number of magical items cited up to that page. Each new header for a magical item increases the number of items described in the entry. To find the total magical items detailed in this encyclopedia, look at the final magical item entry in the last volume.

=== Design Notes ===

The items listed in the ENCYCLOPEDIA MAGICA accessory are categorized wherever possible. This may cause some initial confusion until the reader gets used to the format. The summary below should help solve these problems.

Ammunition: There are three basic types of ammunition in the AD&D game: Arrows (including bolts), sling bullets (for sling weapons), and shot (for powder weapons). All ammunition is to be found in one of these three categories.

Armor: All armor (but not shields, helmets, or gauntlets) is placed in a single category. Random selection charts augment the unique nature of these items.

Books: Magical books have been categorized in five different sections: Book, Libram, Manual, Spellbook, and Tome. Books are used exclusively by priests and Jibrams only by wizards. Manuals can be used by any character class and tomes are magical volumes that can be used by thieves or fighters. Spellbooks contain spells for both wizards and priests.

Musical Instruments: Instruments are separated into three basic types: Wind, percussion, and stringed. In general, if you are looking for a harp, guitar, or harpsichord, look under Stringed Instrument. To find bagpipes, flutes, organs, trumpets, and the like, refer to Wind Instrument. To locate drums, rattles, chimes, bells, or other noisemakers, look under Percussion Instrument.

Potions vs. Oils: A great deal of confusion exists about the difference between potions and oils. Potions, philters, and elixirs are things that must be swallowed to gain the magical effect or property they contain. Oils, salves, and ointments are liquids that are rubbed into the flesh or hair, or coated on items to release their magical properties.

Weapons: There are so many different types of weapons (and many of them have but a single magical version), it was necessary to categorize these as well. The 18 sections on weapons in this encyclopedia are: Axe, Bow (including crossbows), Club, Dagger, Dart, Flail Weapon, Hammer, Lance, Mace, Mattock, Polearm, Powder Weapon, Sickle, Sling, Spear, Staff, Sword, Throwing Star, and Whip. (Quarterstaves and similar weapons are included with magical staves.) To find your favorite weapon, please refer to the index. At the top of each weapon category is a random chart that allows the DM to create unique, magical items that exist only in his or her campaign. For instance, in the DUNGEON MASTER Guide, the holy avenger is always a long sword. Using the charts, it is possible to have a holy avenger khopesh, or even a ninja-to; the possibilities are almost endless.

Most items, clothing, block and tackle, cloaks, wands, staves, etc., are to be found in their usual position in the alphabetical listings. Again, if you have problems finding a particular item, please refer to the index in the last volume. The index is your guide to exciting, magic-laden campaigns.

While this game accessory is intended to be used with the AD&D game system, a little tinkering is all that is required to make the items listed here compatible with traditional D&D game campaigns. The DUNGEONS & DRAGONS Rules Cyclopedia, Appendix Two contains rules that will make conversions easy.

=== Navigating the Encyclopedia ===

We tried to make using this encyclopedia as simple and as easy as possible, and we developed our own system of classification. First, all common items, amulets, books, cloaks, daggers, have their own combined entries. This was done to avoid the boredom of reading “Sword of, Sword of, Sword of,” and to make it easier for you to find the items you need quickly. Within the multiple listings you will find a header, Amulet, for example, followed by alphabetical listings of the amulets. These are listed simply as “ofthe Abyss, Against Disease, of Amiability,” and so on. If the object type was preceded by a title or a name, Cartographer’s, for example, it is listed with just the name of the object—we thought “Attacks Upon the Owner, Jewel of” was a bit unwieldy.

When searching for unique items that are not covered by the group entries, look under an item’s name, not under the subject. For example, you will find Queen Ehlissa’s Marvelous Nightingale under Q, listed just that way—znot as “Nightingale, Queen Ehlissa’s Marvelous.”

=== References ===

In a set of volumes that compiles over 19 years worth of material, there is little space leftover. It was not possible to reveal and detail all of the people and places that appear in the text. Entire modules and boxed sets have been dedicated to them—besides, a little mystery about magical items is a good thing. For the truly curious (and for those of you who want to make sure this encyclopedia is complete), we have placed the name of the TSR product in which the item first appeared (or the one with the most details, if there was more than one use) beneath the GP/XP values of each item. You can research the background of the item, or go creative. You might use the information given as a starting point and tell your own tales. The choice is yours.

=== Rules, Stats, and Monsters ===

Twenty years have elapsed since some of these items have appeared. It was necessary, therefore, to convert all game statistics to AD&D 1nd Edition rules. The D&D Rules Cyclopedia has a chapter on conversion, for those of you who need it, and First Edition DMs and players should be well used to tweaking 2nd Edition rules and stats.

Some of the items from earlier products were altered, names of monsters were brought into line with current practices, and the names of the planes were altered to avoid confusion with the new PLANESCAPE™ line—Tarterus, for example, is now Carceri. Some things could not be changed, however. Athena ’ Shield is still Athena’ Shield. Wherever possible, we have tried to keep the flavor and integrity of the earlier works intact while bringing older material into a new and more enjoyable light.

=== Monetary System Conversion ===

Many campaign worlds under the AD&D banner do not use the same gold piece standard. The DRAGONLANCE® campaign world, DARK SUN® adventure setting, and the Oriental Adventures realm (found in the Kara-Tur boxed set and the Oriental Adventures rulebook) are just three examples. Therefore, before an item can be used (purchased, stolen, or traded), you must convert the value of the item into the monetary system of the game world in which your campaign is set.

In a DRAGONLANCE campaign, the gold piece is either devalued by a factor of 9 or has been replaced by the steel piece (stl). To change this into the proper currency, the DM should either convert the prices given here to steel pieces, or simply multiply the gp value by 10 and keep prices in gold. In the Oriental Adventures realm, the gold piece is not even a viable coin. Therefore, assume the characters must pay in Ch’ao or Tael and multiply the value by two.

Please note that the tables on the following page allow easy conversion of the different coinage found throughout TSR’s game worlds.

Note—The abbreviations used in these volumes are:

DMG = DUNGEON MASTER Guide
PHB = Player’s Handbook

| | ADVANCED DUNGEONS & Dragons Game ||||| | Oriental Adventures |||||
| | PP | GP | EP | SP | CP | | Ch’ien | Ch’ao | Tael | Yuan | Fen |
| AD&D Game ||||||||||||
| Platinum = | 0 | 5 | 10 | 50 | 500 | | 1 | 10 | 10 | 200 | 1000 |
| Gold = | 0/5 | 1 | 2 | 10 | 100 | | 1/5 | 2 | 2 | 40 | 200 |
| Electrum = | 0/10 | 1/2 | 1 | 5 | 50 | | 1/10 | 1 | 1 | 20 | 100 |
| Silver = | 0/50 | 1/10 | 1/5 | 1 | 10 | | 1/50 | 1/5 | 1/5 | 4 | 20 |
| Copper = | 0/500 | 1/100 | 1/50 | 1/10 | 1 | | 1/500 | 1/50 | 1/50 | 1/4 | 2 |
| Oriental Adventures Campaign ||||||||||||
| Ch’len = | 0 | 5 | 10 | 50 | 500 | | 1 | 10 | 10 | 200 | 1000 |
| Ch’ao = | 0/10 | 1/2 | 1 | 5 | 50 | | 1/10 | 1 | 1 | 20 | 100 |
| Tael = | 0/10 | 1/2 | 1 | 5 | 50 | | 1/10 | 1 | 1 | 20 | 100 |
| Yuan = | 0/200 | 1/40 | 1/20 | 1/4 | 4 | | 1/200 | 1/20 | 1/20 | 1 | 5 |
| Fen = | 0/1000 | 1/200 | 1/100 | 1/20 | 1/2 | | 1/1000 | 1/100 | 1/100 | 1/5 | 1 |
| DRAGONLANCE Campaign ||||||||||||
| Platinum = | 0 | 5 | 10 | 50 | 500 | | 1 | 10 | 10 | 200 | 1000 |
| Steel = | 0/5 | 1 | 2 | 10 | 100 | | 1/5 | 5 | 5 | 100 | 500 |
| Iron = | 0/10 | 1/2 | 1 | 5 | 50 | | 1/10 | 1 | 1 | 20 | 100 |
| Bronze = | 0/25 | 1/5 | 2.5 | 2 | 20 | | 1/25 | 1/2.5 | 2.5 | 8 | 40 |
| Gold = | 0/50 | 1/10 | 1/5 | 1 | 10 | | 1/50 | 1/5 | 1/5 | 4 | 20 |
| Silver = | 0/100 | 1/20 | 1/10 | 1/2 | 5 | | 1/100 | 1/10 | 1/10 | 2 | 10 |
| Copper = | 0/500 | 1/100 | 1/50 | 1/10 | 1 | | 1/500 | 1/50 | 1/50 | 1/100 | 1/2 |
| Dark Sun Campaign ||||||||||||
| Platinum = | 0 | 5 | 10 | 50 | 500 | | 1 | 10 | 10 | 200 | 1000 |
| Gold = | 0/5 | 1 | 2 | 10 | 100 | | 1/5 | 2 | 2 | 40 | 200 |
| Electrum = | 0/10 | 1/2 | 1 | 5 | 50 | | 1/10 | 1 | 1 | 20 | 100 |
| Silver = | 0/50 | 1/10 | 1/5 | 1 | 10 | | 1/50 | 1/5 | 1/5 | 4 | 20 |
| Ceramic = | 0/500 | 1/100 | 1/50 | 1/10 | 1 | | 1/500 | 1/50 | 1/50 | 1/4 | 2 |
| Bit = | 0/5000 | 1/100 | 1/500 | 1/100 | 1/10 | | 1/5000 | 1/500 | 1/500 | 1/40 | 1/2 |

| DRAGONLANCE ADVENTURES ||||||| | Dark Sun Campaign ||||||
| |Stl | IP | BP | GP | SP | CP | | PP | GP | EP | SP | CP | Bit |
| AD&D Game ||||||||||||||
| Platinum = | 4 | 10 | 25 | 50 | 100 | 500 | | 1 | 5 | 10 | 50 | 500 | 5000 |
| Gold = | 0 | 2 | 5 | 10 | 20 | 100 | | 1/5 | 1 | 2 | 10 | 100 | 1000 |
| Electrum = | 0/2 | 1 | 2.5 | 5 | 10 | 50 | | 1/10 | 1/2 | 1 | 5 | 50 | 500 |
| Silver = | 0/10 | 1/5 | 1/2 | 1 | 2 | 10 | | 1/50 | 1/10 | 1/5 | 1 | 10 | 100 |
| Copper = | 0/100 | 1/50 | 1/20 | 1/10 | 1/5 | 1 | | 1/500 | 1/100 | 1/50 | 1/10 | 1 | 10 |
| Oriental Adventures Campaign ||||||||||||||
| Ch’ien = | 4 | 10 | 25 | 50 | 100 | 4500 | | 1 | 5 | 10 | 50 | 500 | 5000 |
| Ch’ao = | 0/5 | 1 | 2.5 | 5 | 10 | 50 | | 1/10 | 1/2 | 1 | 5 | 50 | 500 |
| Tael = | 0/5 | 1 | 2.5 | 5 | 10 | 50 | | 1/10 | 1/2 | 1 | 5 | 50 | 500 |
| Yuan = | 0/100 | 1/20 | 1/8 | 1/4 | 1/2 | 10 | | 1/200 | 1/40 | 1/20 | 1/4 | 4 | 40 |
| Fen = | 0/500 | 1/100 | 140 | 1/20 | 1/10 | 2 | | 1/1000 | 1/200 | 1/100 | 1/20 | 1/2 | 2 |
| DRAGONLANCE Campaign ||||||||||||||
| Platinum = | 4 | 10 | 25 | 50 | 100 | 500 | | 1 | 5 | 10 | 50 | 500 | 5000 |
| Steel = | 0 | 2 | 5 | 10 | 20 | 100 | | 1/5 | 1 | 2 | 10 | 100 | 1000 |
| Iron = | 0/2 | 1 | 25 | 5 | 10 | 50 | | 1/10 | 1/2 | 1 | 5 | 50 | 500 |
| Bronze = | 0/5 | 1/2.5 | 1 | 2 | 4 | 20 | | 1/25 | 1/5 | 1/2.5 | 2 | 20 | 200 |
| Gold = | 0/10 | 1/5 | 1/2 | 1 | 2 | 10 | | 1/50 | 1/10 | 1/5 | 1 | 10 | 100 |
| Silver = | 0/20 | 1/10 | 1/4 | 1/2 | 1 | 5 | | 1/100 | 1/20 | 1/10 | 1/2 | 5 | 50 |
| Copper = | 0/100 | 1/50 | 1/20 | 1/10 | 1/5 | 1 | | 1/500 | 1/100 | 1/50 | 1/10 | 1 | 10 |
| Dark Sun Campaign ||||||||||||||
| Platinum = | 4 | 10 | 25 | 50 | 100 | 500 | | 1 | 5 | 10 | 50 | 500 | 5000 |
| Gold = | 0 | 2 | 5 | 10 | 20 | 100 | | 1/5 | 1 | 2 | 10 | 100 | 1000 |
| Electrum = | 0/2 | 1 | 2.5 | 5 | 10 | 50 | | 1/10 | 1/2 | 1 | 5 | 50 | 500 |
| Silver = | 0/10 | 1/5 | 1/2 | 1 | 2 | 10 | | 1/50 | 1/10 | 1/5 | 1 | 10 | 100 |
| Ceramic = | 0/100 | 1/50 | 1/20 | 1/10  | 1/5 | 1 | | 1/500 | 1/100 | 1/50 | 1/10 | 1 | 10 |
| Bit = | 0/1000 | 1/500 | 1/200 | 1/100 | 1/50 | 1/10 | | 1/5000 | 1/1000 | 1/500 | 1/100 | 1/10 | 1 |